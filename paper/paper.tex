\documentclass{vldb}
\usepackage{graphicx}
\usepackage{balance}
\usepackage{url}
\usepackage{amssymb}

% COMANDO PER EVIDENZIARE PARTI DA LEGGERE ATTENTAMETE
\usepackage{color,comment}
\usepackage{soul}
\definecolor{lightcyan}{RGB}{210, 210, 250}
\newcommand{\hlc}[2][lightcyan]{{\sethlcolor{#1}\hl{#2}}}
%\newcommand{\hlc}{}


\begin{document}

\title{E1: ENRON Sentiment}


\numberofauthors{2} 
\author{
\alignauthor
Andrea Jemmett\\
       \affaddr{Vrije Universiteit Amsterdam}\\
       \affaddr{Amsterdam, the Netherlands}\\
       \email{andreajemmett@gmail.com}
\alignauthor
Enrico Rotundo\\
       \affaddr{Vrije Universiteit Amsterdam}\\
       \affaddr{Amsterdam, the Netherlands}\\
       \email{enrico.rotundo@gmail.com}
}


\maketitle

\begin{abstract}
Email dataset analysis is a challenging task in terms of quantity and poor-structured data.
Anyway, the availability of big computational infrastructures such as cluster computers helps to face the former issue.
Indeed, such platforms provide high and scalable computing and unload the programmer from the burden of managing most of its parallelisation and distribution.
Unfortunately, email datasets usually come as unstructured dataset in the form of text files or, whenever they contain any markup structure, the actual data might not be well formed.
In that case, the data could be human-readable but hardly parsable by a machine.
Therefore, the analysis should include additional mining steps and many integrity checks, in order to minimise any possible inconsistencies.  

In the past years, several email datasets from diverse sources have been publicly released.
In this paper, we analyse the famous ``ENRON Corpus'' which contains 620k messages in about 150 mailboxes belonging to ENRON employees involved in a court case.
We extract and analyse sentiments within those messages using functional programming together with a well known engine for large-scale data processing. 
Thus, the analysis is run in a high performance computing cluster.   
We present our result as an interactive visualisation of the sentiment spread via emails together with the company's stock price of the same period.
\hlc{TODO: aggiungere le conclusioni!!!!!!!!!!!}

\end{abstract}


\section{Introduction}
Email is, at least on the user side, a simple mean of communication.
Its popularity is probably due to the simplicity of usage: users can send textual messages and attachments to other addresses, also from mobile devices~\cite{chen2002enterprise}.   
Thus, in the digital era it became very a popular way of communication between privates and companies.
Normally, corporate emails are characterised by a specific structure, for example \textit{user@company.com}, where the \textit{user} suffix is a mailbox identifier and \textit{company.com} is a distinguishable company web domain.
A corporate mailbox server can handle and store thousands of inbound or outbound messages every day, collecting quite a huge amount of exchanged data.

Email dataset analysis consists in analyse a dumped data in order to extract specific information (e.g., communication patterns, sentiment analysis, etc.).
Such analysis is expensive in terms of computation: the data is often composed of a multitude of items that have to be processed individually.
Therefore, such tasks are normally run in distributed environments which allow high degrees of parallelisation.
Cluster computing provides a platform for executing complex parallel tasks in a programmer-friendly environment~\cite{buyya1999high, zaharia2010spark, shvachko2010hadoop}.
This means the programmer does not explicitly code how to parallelise the computation.
Moreover, such systems rely on distributed file systems which provide large storage capabilities and support for redundancy and distributed accesses~\cite{weil2006ceph}.

Furthermore, email datasets usually come in a semi structured fashion in the sense that the actual data might not be well formed. 
For instance, recipients attributes and email's body can be difficult to parse.
Thus, the analysis should include some validation steps which increases the complexity of the whole analysis process.

In this paper, we present a sentiment analysis on the well-known ``ENRON Corpus'' which contains 619,446 messages over 158 users~\cite{shetty2004enron}.
This dataset has been published by the Federal Energy Regulatory Commission\footnote{\url{http://www.ferc.gov/}} during its early 2000s investigation on ENRON Corp. for bankrupt and fraud.
Although it contains the mailboxes of ENRON's employees which were involved in the court case, the messages include text from many more email addresses, for example personal or even external to the company.
We perform a sentiment analysis on that dataset using the state-of-the-art large-scale data processing tools.
Due to the size of the dataset, about 50GB, we need to parallelise the computation.
Thus, we use a functional programming language which is natively supported by Apache Spark engine.
The latter is deployed in a cluster system which runs the whole computation quickly and in a flexible distributed environment.

Our outcome is a visualisation of the sentiment extracted from employees' emails together with the ENRON's stock price of the same period.
\hlc{TODO: aggiungere le conclusioni!!!!!!!!!!!}

The rest of the paper is organised as follow.
Section~\ref{sec:r-w} introduces similar works and the kind of technology used in our work.
Section~\ref{sec:r-q} points out some research questions we try to answer by our analysis.
Section~\ref{sec:p-s} details our analysis setting with respect to the analysis pipeline and its technical architecture. 
Finally, Section~\ref{sec:exp} describes the of experiment run in order to collect our results and Section~\ref{sec:concl} draws some conclusion on the whole work.



\section{Related work}
\label{sec:r-w}

\subsection{Email Dataset Analysis}
About email dataset analysis, literature reports many works focused on exploring, filtering and describing email datasets.
Datasets can be noisy and some preparatory work like filtering and reorganising might be helpful to have a better grip on the data.
For instance,~\cite{klimt2004introducing} provides metrics of the ``ENRON corpus'' as well as a description of it structure.
A thorough analysis of such structure highlights the presence of redundant and SPAM messages.
Similarly,~\cite{zhou2007strategies} describes some cleaning strategies for the aforementioned corpus.
In particular, the authors analyse the actual difficulties in cleaning a corporate email dataset which in the ENRON case are multiple and mainly related to the text-parsing phase.
Indeed, the authors claim there are a certain amount of duplicate emails, addresses and attachments which might come in a slightly different format, making the parsing more challenging.
For example, it is possible to identify duplicate messages by checking the MD5 digest of the email's body constrained by same day~\cite{corrada2004enron}.
Moreover, email bodies ofter report forwarded text or signatures which are not useful for a sentiment extraction.
The authors claim that within the ENRON dataset, only 250k messages are actually unique and they belong to a total of 149 employees.
In~\cite{klimt2004enron}, the authors investigate the feasibility of email folder prediction considering recipients attributes (e.g., \textit{From}, \textit{To}) as well as \textit{Subject} and \textit{body}.
Unfortunately, the F1-score achieved using a Support Vector Machine (SVM) seems very poor, ranging from $0.3$ to $0.7$. 


\subsection{Sentiment Extraction}
Sentiment extraction from text has been well studied in the past 10 years~\cite{aggarwal2012mining, das2007yahoo, bai2004sentiment, gamon2005pulse, bird2009natural}.
In~\cite{manning2014stanford}, the authors present a powerful deep-learning based tool for text annotation which provides sentiment labelling on a sentence-grain.
That tool is the state-of-the-art in text annotation and is distributed as a fast and easy-to-use Open Source library.
A live demo is also available on the related website\footnote{\url{http://nlp.stanford.edu:8080/sentiment/rntnDemo.html}}.
Most emails contain human written text, therefore it is likely to contain some kind of emotions.
Its spread is influenced by many factors (e.g., social, behavioural, etc.).
For example,~\cite{mohammad2011tracking} shows emotion patterns in email messages occur with different characteristics depending on the genders involved.
Specifically, the authors consider the eight basic and prototypical emotions~\cite{plutchik1980emotion} and point out their balance is biased depending on the gender of the sender/receiver genders.


\subsection{Large-scale Data Processing Tools}
TODO


\section{Research questions}
\label{sec:r-q}
TODO

\section{Project setup}
\label{sec:p-s}
TODO

\section{Experiments}
\label{sec:exp}
TODO

\section{Conclusions}
\label{sec:concl}
TODO

\section{Acknowledgments}
SURFsara maybe?...

\clearpage
\balance
\bibliographystyle{abbrv}
\bibliography{bibliography} 

\begin{appendix}
TODO

\section{BLA BLA}
TODO

\end{appendix}



\end{document}
